\documentclass[10pt,a4paper]{article}
\usepackage[utf8]{inputenc}
\usepackage[spanish]{babel}
\usepackage{amsmath}
\usepackage{amsfonts}
\usepackage{amssymb}
\usepackage{makeidx}
\usepackage{graphicx}
\usepackage[left=2cm,right=2cm,top=2cm,bottom=2cm]{geometry}
\author{Universidad de Guayaquil, Giampaolo Delgado, Anthony Lozano, Kevin Ortiz}
\title{Documentación de diseño y desarrollo de TestDisk}
\begin{document}

\maketitle

\section{para el proceso de diseño y desarrollo}
\subsection{Requerimientos}
(Se cumple la norma ISO 9001-7.3.3 con todos sus incisos pues las entradas del proceso las cuales son los requerimientos y especificaciones se expresan de forma clara y sin ambigüedad)\\

Se busca crear una potente aplicación enfocada para usuarios técnicos y administradores de equipos de cómputo que se encargue de:
\begin{itemize}
\item[•]Arreglar la Tabla de Particiones, recuperar particiones eliminadas
\item[•]Recuperar sectores de arranque FAT32 de su copia de seguridad
\item[•]Reconstruir sectores de arranque FAT12/FAT16/FAT32
\item[•]Arreglar tablas de arranque de tipo FAT
\item[•]Reconstruir sectores de arranque NTFS
\item[•]Recuperar sectores de arranque NTFS de su copia de seguridad
\item[•]Arreglas la MFT usando la MFT imagen
\item[•]Localizar el Superblock de copia de seguridad de ext2/ext3/ext4
\item[•]Recuperar archivos del sistema de archivos FAT, NTFS y ext2
\item[•]Copiar archivos de particiones FAT, NTFS y ext2/ext3/ext4 eliminadas
\item[•]Poder trabajar en máquinas remotas a con interoperabilidad entre sistemas operativos 
\item[•]Compatibilidad con tecnologías de almacenamiento redundante de discos.
\end{itemize}
Se necesita recuperar archivos de varios discos duros con diferentes sistemas de archivos para dar cobertura a las empresas que implementen TaskDisk en su red de trabajo, estos son los sistemas de archivos compatibles:
\begin{itemize}
\item[•]BeFS (BeOS)
\item[•]BSD disklabel (FreeBSD/OpenBSD/NetBSD)
\item[•]CramFS, Sistema de Archivos Comprimidos
\item[•]DOS/Windows FAT12, FAT16 y FAT32
\item[•]Windows exFAT
\item[•]HFS and HFS+, Sistema de Archivos Jerárquicos
\item[•]JFS, IBM's Sistema Diario de Archivos(Journaled File System)
\item[•]Linux ext2, ext3 y ext4
\item[•]Linux Raid
\begin{itemize}
\item[•]RAID 1: espejos
\item[•]RAID 4: arreglos con dispositivo de paridad rayados
\item[•]RAID 5: arregloscon información de paridad distribuida rayados
\item[•]RAID 6: arreglos con información de dúo redundancia distribuida rayados
\end{itemize}
\item[•]Linux Swap (versiones 1 y 2)
\item[•]LVM y LVM2, Administración de Volumen Lógico de Linux
\item[•]Mapa de particiones Mac
\item[•]NSS Servicio de Almacenamiento Novell
\item[•]NTFS ( Windows NT/2K/XP/2003/Vista )
\item[•]ReiserFS 3.5, 3.6 y 4
\item[•]Sun Solaris i386 disklabel
\item[•]UFS y UFS2, Sistema de Archivos Unix (Sun/BSD/...)
\item[•]XFS, SGI's Sistema Diario de Archivos(Journaled File System)
\end{itemize}
\subsection{Modelo de negocios}
La distribución de TaskDisk  es gratuita y libre bajo licencia GNU su modificación y contribución son libres para usuarios desarrolladores GNU sujeto a permisos especiales de licencia, los usuarios que deseen contribuir con los desarrolladores con donaciones económicas, pueden hacerlo desde la página oficial del programa. \\

(Se cumple la norma ISO 9001-7.3.2 inciso “b” ya que se da la información legal sobre el uso de la aplicación)
\subsection{Autores responsables de la aplicación}
 TestDisk está escrito y mantenido por Christophe GRENIER <grenier@cgsecurity.org>\\
Los logotipos de TestDisk fueron creados por Simone Brandt y Dmitri Zdorov en 2001.\\
TestDisk es producido por la organización Cgsecurity.\\

(Se cumple la norma ISO 9001-7.3.1 inciso “c” ya que se definen los responsables del proyecto)
\section{Diseño}
(Se cumple la norma ISO 9001-7.3.1 inciso “a” ya que se define el proceso de diseño y desarrollo)
\subsection{Herramientas}
\subsubsection{Lenguaje}
Debido a que se trata de una aplicación que va a tratar directamente con hardware se debe utilizar un lenguaje que se compile directamente a binario y tanga un menor coeficiente de relación código-instrucción por lo cual el lenguaje elegido para la codificación del proyecto fue C/C++.\\
\subsubsection{Herramientas de compilación}
Debido a que la aplicación está diseñada como software libre de código abierto también se comparten las herramienta de compilación para los diferentes sistemas operativos soportados:
\\

DOS: 
TestDisk y PhotoRec pueden compilar desde DOS/Win9x usando DJGPP. Leer DOS para más información.\\

Mac OS X: 
Para compilar TestDisk y PhotoRec, instalar Apple Development Kit desde el DVD de instalación de Mac-OS X. Ver Mac OS X para notas adiccionales.\\

Linux:
\begin{itemize}
\item Debian, Ubuntu: apt-get, install, build-essential, e2fslibs-dev, libncurses5-dev, libncursesw5-dev, libntfs-dev, libjpeg62-dev, libreiserfs0.3-dev, uuid-dev, zlib1g-dev
\item Fedora, RedHat Enterprise, CentOS: yum, install, ncurses-devel, e2fsprogs-devel, libjpeg-devel, ntfsprogs-devel, libewf-devel, zlib-devel, openssl-devel, libuuid-devel. Son necesarios paquetes adiccionales para la versión instalada (esta versión no usa las bibliotecas locales): ncurses-static libjpeg-static zlib-static openssl-static glibc-static.
\item Mandriva: urpmi, ncurses-devel, e2fsprogs-devel, jpeg-devel, libntfs-devel, zlib-devel, openssl-devel, libewf-devel
\item OpenSuse: install, ncurses-devel, e2fsprogs-devel, libjpeg-devel, ntfsprogs-devel, zlib-devel, openssl-devel, libreiserfs-devel, libuuid-devel
\end{itemize}
Windows:
Usar el compilador libre C/C++ Cygwin para realizar la version de Windows (Windows NT 4/Windows 2000/XP/2003/Vista...).
También es posible usando el compilador MinGW pero algunas funcionalidades/Bibliotecas pueden estar perdidas. Para detalles adiccionales leer Compile Win.
\subsection{Interfaz}
Al ser una aplicación destinada para usuarios técnicos y estar pensado como herramienta open source , esta funciona a través de la terminal de comandos del sistema operativo siendo una TUI(text user interface).
\end{document}